\documentclass[a4paper,10pt]{article}
\usepackage[left=1in, right=1in, top=1in, bottom=1in]{geometry}
\usepackage{titlesec}
\usepackage{enumitem}
\usepackage{paralist}
\usepackage{hyperref}
\usepackage{xcolor}
\usepackage{helvet}
\usepackage{mathptmx} % Times New Roman
\titleformat{\section}{\large\bfseries}{\thesection}{1em}{}[\titlerule]
\titleformat{\subsection}{\bfseries}{\thesubsection}{1em}{}

\begin{document}

% Header Section
\begin{center}
    {\huge \textbf{ Ganesh }} \\
    \vspace{5pt}
    \small \href{mailto:pujithganesh18@gmail.com } pujithganesh18@gmail.com \quad \textbullet{} 7661869925 \quad  \quad \textbullet{} \href{ https://www.linkedin.com/in/ganesh10-/ }{linkedin.com} \quad \textbullet{} \href{  }{github.com}
\end{center}

% Education Section
\section*{Education}

\noindent
\textbf{  } \\
\textit{ B.E. } \hfill \textit{  } \\

\vspace{0.3cm}

\noindent
\textbf{  } \\
\textit{ B.E. } \hfill \textit{  } \\

\vspace{0.3cm}

\noindent
\textbf{  } \\
\textit{ B.E. } \hfill \textit{  } \\

\vspace{0.3cm}

\noindent
\textbf{  } \\
\textit{ B.E. } \hfill \textit{  } \\

\vspace{0.3cm}


% Projects Section

\section*{Projects}

\noindent
\textbf{ Solar System Stimulation } \hfill \textit{Tech Stack: N/A } \\
\vspace{5pt} \textit{ N/A }
\begin{compactitem}
    \item 
\end{compactitem}



% Skills Section

\section*{Skills}
\noindent
\begin{compactitem}
    
        \item MERN Stack 
    
        \item Java 
    
        \item Python
    
\end{compactitem}


% Certifications Section

\section*{Certifications}
\noindent
\begin{compactitem}
    
        \item MongoDB Associate Developer \hfill \textit{ N/A }
    
        \item SalesForce AI Associate \hfill \textit{ N/A }
    
\end{compactitem}


\end{document}